\documentclass[12pt, a4paper]{report}
\usepackage[T1]{fontenc}
\usepackage[utf8]{inputenc}
\usepackage[french]{babel}
\usepackage{chemist}
\usepackage{graphicx}
\usepackage{caption}

\begin{document}


\vspace{5cm}

\begin{center}

\begin{LARGE}
\textbf{Projet Informatique : "DJ'Oz"}
\end{LARGE}
	
\vspace{1cm}

\begin{Large}
\begin{center}
LFSAB1402 - Projet 2014
\vspace{0.5cm}
\end{center}


Groupe 80 

\vspace{0.5cm}
de Bellefroid Cédric - 32631100 \\

\& \\
de Bellefroid Martin - 96091300
\end{Large}

\end{center}
\newpage

\chapter{Structure du programme}
	Dans ce chapitre, nous pouvons retrouver les explications concernant l'implémentation des fonctions \textit{fun \{Interprete Partition\} } et \textit{fun \{Mix Interprete Music\} }, les explications concernant l'implémentation des fonctions secondaires ainsi que les décisions prises quant à la conception de celles-ci.
	
\section{Fonction "\textit{fun \{Interprete Partition\} }"}


\section{Fonction "\textit{fun \{Mix Interprete Music\} }"}


\section{Fonctions secondaires}


\chapter{Difficultés rencontrées}
	Nous pouvons retrouver ici les difficultés rencontrées lors de l'implémentation des fonctions demandées et de certaines fonctions secondaires utilisées ainsi que la façon dont nous avons géré ces complications.
	
\section{Fonction "\textit{fun \{Interprete Partition\} }"}


\section{Fonction "\textit{fun \{Mix Interprete Music\} }"}


\section{Fonctions secondaires}


\chapter{Limitations et problèmes connus}
	Ce chapitre reprend toutes les lacunes que nous avons décelées dans notre programme. 



\chapter{Extensions}
	Ce chapitre reprend les différentes extensions que nous avons implémentées visant à améliorer, étendre et mettre en oeuvre des aspects spécifiques de notre implémentation.


\end{document}

